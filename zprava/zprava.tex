\documentclass{article}
\usepackage{minted}
\usepackage{cprotect}
\usepackage{hyperref}
\usepackage{geometry}
\oddsidemargin=-5mm
\evensidemargin=-5mm\marginparwidth=.08in \marginparsep=.01in
\marginparpush=5pt\topmargin=-15mm\headheight=12pt
\headsep=25pt
%\footheight=12pt
\footskip=30pt\textheight=25cm
\textwidth=17cm\columnsep=2mm
\columnseprule=1pt\parindent=15pt\parskip=2pt
\renewcommand{\figurename}{Obrázek}
\usepackage{fancyvrb}
\usepackage{listings}
\usepackage[czech]{babel}
\newcommand{\heuristika}[1]{\textbf{\textit{Heuristika {#1}.}}}
\newcommand{\const}[1]{\mintinline{bash}{#1}}
\newcommand{\konstanta}[1]{\textbf{\textit{Konstanta}} \const{#1}.}

% minted centering
\usepackage{xpatch,letltxmacro}
\LetLtxMacro{\cminted}{\minted}
\let\endcminted\endminted
\xpretocmd{\cminted}{\RecustomVerbatimEnvironment{Verbatim}{BVerbatim}{}}{}{}

%%%%%%%%%%%%%%%%%%%%%%%%%%%%%%%%%%%%%%%%%%%%%%%%%%%%%%%%%%%%%%%%%%%%%%%%%%%%%%%
%http://ftp.cvut.cz/tex-archive/macros/latex/contrib/algorithm2e/doc/algorithm2e.pdf
%\usepackage[ruled,vlined]{algorithm2e}
\usepackage[linesnumbered,lined,boxed,commentsnumbered]{algorithm2e}
%%%%%%%%%%%%%%%%%%%%%%%%%%%%%%%%%%%%%%%%%%%%%%%%%%%%%%%%%%%%%%%%%%%%%%%%%%%%%%%

\usepackage{multirow}

\begin{document}
    \begin{center}
        \bf Semestrální projekt MI-PDP 2020/2021:\\[5mm]
        Paralelní algoritmus pro řesení problému prohledávání stavového prostoru\\[5mm]
        Martin Šafránek\\[2mm]
        magisterské studijum, FIT ČVUT, Thákurova 9, 160 00 Praha 6\\[2mm]
        zdrojové kódy:\\ \url{https://github.com/TaIos/ni-pdp-semestralka}\\[2mm]
        \today
    \end{center}


    \label{sec:sequential}


    \section{Definice problému a popis sekvenčního algoritmu}

    Program řeší problém nalezení optimální posloupnosti tahů pro střelce a jezdce, která vede k sebrání
    všech pěšců rozmístěných na šachovnici. Jedná se o analogii problému obchodního cestujícího.
    Nalezení optimálního řešení je proto NP těžký úkol. Řešení v této práci používá bruteforce
    s heuristikami pro ořezávání stavového prostoru.
    \newline

    \label{subsec:popis-vstupu}

    \subsection{Popis vstupu}
    Příklad vstupu je uveden na obrázku \ref{fig:board-example}. Obsahuje popořadě vždy
    \begin{enumerate}
        \item přirozené číslo $k$, reprezentující délku strany šachovnice $S$ o velikosti $k \times k$,
        \item horní mez délky optimální posloupnosti $d^{*}_{max}$,
        \item pole souřadnic rozmístěných figurek na šachovnici S.
    \end{enumerate}

    \begin{figure}[h]
        \begin{center}
            \begin{BVerbatim}
                11
                22
                -----------
                -------P---
                P-----P--P-
                ---PJS--P--
                ----P------
                ---P--P----
                ------P----
                ------P----
                ------P----
                -----PP----
                -----------
            \end{BVerbatim}
        \end{center}
        \cprotect\caption{ Příklad vstupních dat pro $k=11$, $d^{*}_{max}=22$. Střelec je označen \verb|S|, jezdec \verb|J|,\
            pěšák \verb|P| a prázdné políčko \verb|-|. }
        \label{fig:board-example}
    \end{figure}


    \label{subsec:seq-heuristics}

    \subsection{Heuristiky}

    Sekvenční algoritmus používá dvě heuristiky pro pohyb střelce a koně.

    \heuristika{střelec} Z množiny možných políček, kam je možné střelce přemístit jsou preferována ty,
    která obsahují pěšce. Pokud takové políčko neexistuje, jsou preferována políčka s alespoň jedním pěšákem
    na diagonále. Jinak se pohyb střelce rozhodne náhodně.
    \hspace{1.5pt}

    \heuristika{kůň} Z množiny možných políček, kam je možné koně přemístit jsou preferována ty,
    která obsahují pěšce. Pokud takové políčko neexistuje, jsou preferována políčka, z kterých
    kůň ve svém následujícím tahu může vzít pěšáka. Pokud ani takové políčko neexistuje, jsou
    preferováno políčka, z kterých kůň v následujícíh dvou tazích může vzít pěšáka.
    Jinak se pohyb koně rozhodne náhodně.

    \subsection{Pseudokód}
    \begin{algorithm}[H]
        \SetAlgoLined
        \DontPrintSemicolon
        \SetAlgorithmName{Algoritmus}{}{}
        \SetKwFunction{betterSolutionExists}{existujeLepšíŘešení}
        \SetKwFunction{updateBestSolution}{aktualizujNejlepšíŘešení}
        \SetKwFunction{HeuristikaKun}{HeuristikaKůň}
        \SetKwFunction{HeuristikaStrelec}{HeuristikaStřelec}
        \SetKwFunction{MakeMove}{ProveďTah}
        \SetKwFunction{FBetterSolutionExists}{existujeLepšíŘešení}

        \SetKwProg{Fn}{Function}{:}{}
        \Fn{\FBetterSolutionExists{šachovnice $S$, nejlepší řešení $S^*$}}{
        \uIf{počet pěšáků $S$ + počet tahů $S$ $\ge$ délka $S^*$
            \newline {\bf or}
            počet tahů $S$ + počet pěšáků $S$ > maximální hloubka
            \newline {\bf or}
            $S^*$ má minimální možný počet tahů
        } {
                \Return {\bf True}\;
            } \Else{
            \Return {\bf False}\;
        }\;
        }\;


        \SetKwFunction{FMain}{sequence}
        \SetKwProg{Fn}{Function}{:}{}
        \Fn{\FMain{šachovnice $S$, nejlepší řešení $S^*$}}{
            \If{\betterSolutionExists{S, S*}} {
                \Return
            }\;
            \If{počet pěšáků $S$ je 0} {
                \updateBestSolution{S, S*}\;
                \Return\;
            }\;

            \SetKwData{Tahy}{Tahy}
            \Tahy$\leftarrow$ prázdný list\;
        \If{je na tahu kůň}{
            \Tahy = \HeuristikaKun{$S$}
        }\;
            \If{je na tahu střelec}{
                \Tahy = \HeuristikaStrelec{$S$}
            }\;

            \SetKwData{Next}{$S_{next}$}
            \ForAll{$t$ in \Tahy} {
                \Next = \MakeMove{$S$, $t$}\;
                \FMain{\Next, $S^*$}\;
            }\;
        }\;
        \caption{sekvenční}
        \label{alg:sequential}
    \end{algorithm}

    \label{sec:task-par}
    \section{Popis paralelního algoritmu a jeho implementace v OpenMP - taskový paralelismus}

    Taskový paralelní algoritmus je naimplementován pomocí OpenMP. Hlavní rozdíl oproti sekvenčnímu algoritmu
    popsaném v \ref{sec:sequential} je rozdělení úlohy prohledávání stavového prostoru na tasky. Task je
    základní jednotka, kterou je OpenMP schopno přidělit vláknu a provést tak výpočet. Pro zadanou úlohu
    task znamená šachovnici s pozicí všech figurek a historií tahů. Takto vytvořené tasky OpenMP přidává
    do svého taskpoolu, z kterého si je vlákna vyzvedávají a řeší. Dále všechny vlákna řešící tasky z
    taskpoolu sdílejí nejlepší řešení $d_{best}$. Heuristiky jsou totožné jako v podsekci \ref{subsec:seq-heuristics}.

    \subsection{Konstanty a parametry pro škálování algoritmu}
    Taskový paralelní algoritmus implementovaný pomocí OpenMP umožňuje nastavení konstant, které ovlivní
    logiku funkce programu a tedy i výpočetní čas. Změněny byly pouze zde zmíněné konstanty. Jejich hodnota
    byla určena empiricky na vstupních datech. Nejedná se o optimální hodnoty, protože jejich nalezení je
    stejně těžký problém jako nalezení optimální cesty v původním problému.

    \konstanta{TASK_THRESHOLD}  Pokud vlákno řeší instanci a délka její cesty je delší než \const{TASK_THRESHOLD}, nevytváří další
    OpenMP tasky a nepřidává je do taskpoolu. Zadanou instaci vyřeší použitím sekvenčního algoritmu popsaném v sekci \ref{sec:sequential}.

    \begin{table}[hb]
        \centering
        \begin{tabular}{|l|l|}
            \hline
            název                  & hodnota \\ \hline
            \const{TASK_THRESHOLD} & 6       \\ \hline
        \end{tabular}
        \caption{Konstanty použité v OpenMP taskovém paralelismu.}
        \label{tab:data-par-constants}
    \end{table}

    \subsection{Pseudokód}

    \begin{algorithm}[H]
        \SetAlgoLined
        \DontPrintSemicolon
        \SetAlgorithmName{Algoritmus}{}{}
        \SetKwFunction{betterSolutionExists}{existujeLepšíŘešení}
        \SetKwFunction{UpdateBestSolution}{aktualizujNejlepšíŘešení}
        \SetKwFunction{HeuristikaKun}{HeuristikaKůň}
        \SetKwFunction{HeuristikaStrelec}{HeuristikaStřelec}
        \SetKwFunction{MakeMove}{ProveďTah}
        \SetKwFunction{FBetterSolutionExists}{existujeLepšíŘešení}

        \SetKwFunction{FMain}{openMpTask}
        \SetKwProg{Fn}{Function}{:}{}
        \Fn{\FMain{šachovnice $S$, nejlepší řešení $S^*$}}{
            \If{\betterSolutionExists{S, S*}}{
                \Return\;
            }\;

        \If{počet pěšáků $S$ je 0} {
            \If{\betterSolutionExists{S, S*}}{
                \texttt{\#pragma omp critical}\;
                \If{\betterSolutionExists{S, S*}}{
                    \UpdateBestSolution{S, S*}\;
                }\;
            }\;
            \Return\;
        }\;

            \SetKwData{Tahy}{Tahy}
            \Tahy$\leftarrow$ prázdný list\;
            \If{je na tahu kůň}{
                \Tahy = \HeuristikaKun{$S$}
            }\;
            \If{je na tahu střelec}{
                \Tahy = \HeuristikaStrelec{$S$}
            }\;

            \SetKwData{Next}{$S_{next}$}
            \ForAll{$t$ in \Tahy} {
                \Next = \MakeMove{$S$, $t$}\;
                \texttt{\#pragma omp task firstprivate($\ldots$)}\;
                \FMain{\Next, $S^*$}\;
            }\;
        }\;
        \caption{OpenMP task}
        \label{alg:openmp-task}
    \end{algorithm}


    \label{sec:data-par}
    \section{Popis paralelniho algoritmu a jeho implementace v OpenMP - datový paralelismus}
    Datový paralelismus v OpenMP pracuje s datově nezávislými celky, které podle určené
    strategie přiděluje vláknům na zpracování. Nezávislý datový celek je pro zadanou úlohu
    šachovnice s pozicí všech figurek a historií tahů.

    První krok je vygenerování datově nezávislých celků – to je provedeno před použitím OpenMP.
    Ty jsou následně najednou předány OpenMP. To je určenou strategií rozdělí mezi vlákna.
    Každé vlákno pak provádí sekvenční řešení problému popsané v sekci \ref{sec:sequential}.
    Vlákna mezi sebou sdílejí pouze nejlepší řešení.

    \subsection{Konstanty a parametry pro škálování algoritmu}
    Prvním krokem před spuštěním OpenMP řešení je vytvoření datově nezávislých instancí.
    Ty se vytvoří použitím sekvečního algoritmu, viz sekce \ref{sec:sequential}.
    Jejich počet je regulován konstantou \const{EPOCH_CNT}.

    Parametrem OpenMP je konstanta \const{schedule}. Ta určuje politiku přidělování datově nezávislých instací vláknům.
    Zde je použitá hodnota \const{dynamic()} bez parametrů. To znamená, že pokud vlákno dokončí výpočet je mu
    přiřazena jedna další datově nezávislá instance k vyřešení.

    \konstanta{EPOCH_CNT} Určuje, kolik datově nezávislých instancí je vygenerováno.
    Pro každou epochu jsou provedeny všechny možné tahy buď koněm, nebo střelcem.
    Po vyčerpání všech epoch jsou stavy, do kterých se kůň a střelec dostali
    použity jako nezávislé instance.

    \konstanta{schedule} OpenMP konstanta, která nastavuje politiku přidělování datově nezávislých instací vláknům.

    \begin{table}[hb]
        \centering
        \begin{tabular}{|l|l|}
            \hline
            název      & hodnota \\ \hline
            \const{EPOCH_CNT} & 3       \\ \hline
            \const{schedule} & dynamic()      \\ \hline
        \end{tabular}
        \caption{Konstanty použité v OpenMP datovém paralelismu}
        \label{tab:data-par-constants}
    \end{table}

    \subsection{Pseudokód paralelního algoritmu — datový paralelismus}
    Porchetta andouille flank kielbasa. Tail biltong turducken porchetta burgdoggen ground round shoulder ham, hamburger bacon shankle landjaeger fatback pork belly doner. Sirloin doner venison shankle cow, hamburger flank sausage pork belly. Tenderloin venison pancetta corned beef tongue cow pork belly capicola ball tip salami short ribs. Sirloin rump andouille tail shank fatback bresaola.

    Leberkas ham bacon, pastrami turducken pork belly cupim salami kielbasa doner. Turkey cupim meatball capicola jowl cow shank chicken drumstick kevin salami swine pork belly. Drumstick leberkas corned beef beef short loin boudin. Turkey strip steak bacon, ball tip sirloin pork loin pork.

    \subsection{Pseudokód}

    \begin{algorithm}[H]
        \SetAlgoLined
        \DontPrintSemicolon
        \SetAlgorithmName{Algoritmus}{}{}
        \SetKwFunction{GenerateInstances}{vygenerujInstance}
        \SetKwFunction{SequenceWithOmpCritical}{sequenceAlgorgitmusSOmpCriticalProUpdate}
        \SetKwData{Instance}{Instance}

        \SetKwFunction{FMain}{openMpData}
        \SetKwProg{Fn}{Function}{:}{}
        \Fn{\FMain{šachovnice $S$, nejlepší řešení $S^*$}}{
            \Instance$\leftarrow$ \GenerateInstances{$S$}\;
            \texttt{\#pragma omp parallel for $\ldots$}\;
            \ForAll{$S_{gen}$ in \Instance} {
            \SequenceWithOmpCritical{$S_{gen}$, $S^*$}
            }\;
        }\;
        \caption{OpenMP data}
        \label{alg:openmp-data}
    \end{algorithm}


    \section{Popis paralelního algoritmu a jeho implementace v MPI}

    Řešení s použitím MPI se skládá ze dvou částí. První je datový OpenMP paralelismus,
    viz sekce \ref{sec:data-par}. Druhou část tvoří MPI. Ten má za úkol řídit a distribuovat
    výpočet na několika výpočetních uzlech.

    MPI proces začíná tím, že si master vlákno identickým způsobem jako v sekci s datovým paralelismem\ref{sec:data-par}
    vygeneruje datově nezávislé instance. Ty pak serializuje a společně s globálním nejlepším řešením je
    pošle přes MPI interface slavům. Každý z nich pomocí datového paralelismu popsaného v sekci \ref{sec:data-par}
    vyřeší přijmuté řešení a odešlě ho zpět master vláknu. Pak požádá master vlákno o další instanci k vyřešení.
    Pokud master vláknu dojdou instance k vyřešení, rozešlě slavům zprávu o ukončení výpočtu.

    \subsection{Konstanty a parametry pro škálování algoritmu}
    Protože MPI využívá pro řešení tasků datový OpenMP paralelismus popsaný v sekci \ref{sec:data-par}, jsou
    zde uvedeny pouze MPI konstanty.

    TODO nastaveni poctu vypocetnich jader

    \subsection{Pseudokód}

    \begin{algorithm}[H]
        \SetAlgoLined
        \DontPrintSemicolon
        \SetAlgorithmName{Algoritmus}{}{}
        \SetKwFunction{GenerateInstances}{vygenerujInstance}
        \SetKwFunction{SequenceWithOmpCritical}{sequenceAlgorgitmusSOmpCriticalProUpdate}
        \SetKwFunction{MpiSend}{MPI\_Send}
        \SetKwFunction{MpiIProbe}{MPI\_Iprobe($\ldots$)}
        \SetKwFunction{MpiRecv}{MPI\_Recv($\ldots$)}
        \SetKwData{Instance}{Instance}
        \SetKwData{InstanceAti}{Instance[i]}
        \SetKwFunction{UpdateBestSolution}{aktualizujNejlepšíŘešení}
        \SetKwFunction{betterSolutionExists}{existujeLepšíŘešení}
        \SetKwFunction{OpenMpData}{openMpData}


        \SetKwFunction{FMain}{mainMPI}
        \SetKwProg{Fn}{Function}{:}{}
        \Fn{\FMain{šachovnice $S$, nejlepší řešení $S^*$}}{
            \uIf{master vlákno} {
                \Instance$\leftarrow$ \GenerateInstances{$S$}\;
                \For{$i \leftarrow 1 \ldots$ počet slave procesů} {
                \MpiSend{$slave_i$, \InstanceAti, $S^*$}\;
                }\;

                \SetKwData{Cnt}{alive}
                \Cnt $\leftarrow$ počet slave procesů\;
            \While{True} {
                \If{\MpiIProbe} {
                    $S_{slave}$, $id_{slave} \leftarrow$ \MpiRecv\;
                    \If{not \betterSolutionExists{$S_{slave}$, $S^*$}} {
                        \UpdateBestSolution{$S_{slave}$, $S^*$}\;
                    }\;

                \uIf{zvýbá nějaká nevyřešená instance $S'$ v \Instance}{
                    \MpiSend{$id_{slave}$, $S'$, $S^*$}
                }
                \Else{
                    \MpiSend{$id_{slave}$, ukončující token}\;
                    \Cnt $\leftarrow$ \Cnt $-$ 1

                }\;

            \If {\Cnt $=0$}{
                    {\bf break}\;
            }\;
            }\;

            }\;

            } \ElseIf{slave vlákno} {
                \While{True} {
                    \If{\MpiIProbe} {
                        $S'$, $S^*, token \leftarrow$ \MpiRecv\;
                        \uIf{$token$ =  práce} {
                        $S_{slave} \leftarrow$ \OpenMpData{$S'$, $S^*$} \;
                        \MpiSend{$id_{master}$, $S_{slave}$}\;
                        } \ElseIf{$token$ =  konec} {
                            {\bf break}\;
                        }\;

                    }\;
                }\;

            }\;
        }\;
        \caption{MPI}
        \label{alg:mpi}
    \end{algorithm}


    \section{Naměřené výsledky a vyhodnocení}

    \begin{enumerate}
        \item Zvolte tri instance problemu s takovou velikosti vstupnich dat, pro ktere ma sekvencni
        algoritmus casovou slozitost mezi 1 a 10 minutami.
        Pro mereni cas potrebny na cteni dat z disku a ulozeni na disk neuvazujte a zakomentujte
        ladici tisky, logy, zpravy a vystupy.
        \item Merte paralelni cas pri pouziti $i=2,\cdot,60$ vypocetnich jader.
        \item Tabulkova a pripadne graficky zpracovane namerene hodnoty casove slozitosti měernych instanci behu programu s popisem instanci dat. Z namerenych dat sestavte grafy zrychleni $S(n,p)$.
        \item Analyza a hodnoceni vlastnosti paralelniho programu, zvlaste jeho efektivnosti a skalovatelnosti, pripadne popis zjisteneho superlinearniho zrychleni.
    \end{enumerate}

    \begin{table}[]
        \centering
        \begin{tabular}{|c|c|}
            \hline
            \multicolumn{1}{|l|}{\textbf{instance}} & \multicolumn{1}{l|}{\textbf{{}doba běhu} {[}m{]}} \\ \hline
            saj7                           & 1.727                                       \\ \hline
            saj10                          & 8.163                                       \\ \hline
            saj12                          & 4.542                                       \\ \hline
        \end{tabular}
        \caption{Sekvenční algoritmus -- doba běhu v minutách.}
        \label{tab:seq}
    \end{table}

    \begin{table}[]
        \centering
        \begin{tabular}{|c|c|c|}
            \hline
            \multicolumn{1}{|l|}{\textbf{instance}} & \multicolumn{1}{l|}{\textbf{\#jader}} & \textbf{doba běhu} {[}m{]} \\ \hline
            \multirow{8}{*}{saj7}  & 1  & 0.251 \\
            & 2  & 0.311 \\
            & 4  & 0.017 \\
            & 6  & 0.075 \\
            & 8  & 0.047 \\
            & 10 & 0.019 \\
            & 16 & 0.004 \\
            & 20 & 0.006 \\ \hline
            \multirow{8}{*}{saj10} & 1  & 2.199 \\
            & 2  & 3.321 \\
            & 4  & 2.077 \\
            & 6  & 1.695 \\
            & 8  & 1.428 \\
            & 10 & 1.078 \\
            & 16 & 1.439 \\
            & 20 & 1.624 \\ \hline
            \multirow{8}{*}{saj12} & 1  & 9.555 \\
            & 2  & $\infty$   \\
            & 4  & 6.303 \\
            & 6  & 8.426 \\
            & 8  & 8.113 \\
            & 10 & 5.946 \\
            & 16 & $\infty$   \\
            & 20 & 5.654 \\ \hline
        \end{tabular}
        \caption{OpenMP task algoritmus -- doba běhu v minutách.}
        \label{tab:openmp-task}
    \end{table}

    % Please add the following required packages to your document preamble:
% \usepackage{multirow}
    \begin{table}[]
        \centering
        \begin{tabular}{|c|c|c|}
            \hline
            \multicolumn{1}{|l|}{\textbf{instance}} & \multicolumn{1}{l|}{\textbf{\#jader}} & \textbf{doba běhu} {[}m{]} \\ \hline
            \multirow{8}{*}{saj7}  & 1  & 5.207 \\
            & 2  & 4.935 \\
            & 4  & 2.283 \\
            & 6  & 0.944 \\
            & 8  & 0.775 \\
            & 10 & 0.005 \\
            & 16 & 0.007 \\
            & 20 & 0.004 \\ \hline
            \multirow{8}{*}{saj10} & 1  & 2.032 \\
            & 2  & 2.948 \\
            & 4  & 2.133 \\
            & 6  & 1.817 \\
            & 8  & 1.667 \\
            & 10 & 1.642 \\
            & 16 & 1.686 \\
            & 20 & 1.368 \\ \hline
            \multirow{8}{*}{saj12} & 1  & 9.610 \\
            & 2  & 8.232 \\
            & 4  & 6.927 \\
            & 6  & 5.880 \\
            & 8  & 6.409 \\
            & 10 & 6.695 \\
            & 16 & 4.788 \\
            & 20 & 8.986 \\ \hline
        \end{tabular}
        \caption{OpenMP data algoritmus -- doba běhu v minutách.}
        \label{tab:openmp-data}
    \end{table}

    % Please add the following required packages to your document preamble:
% \usepackage{multirow}
    \begin{table}[]
        \centering
        \begin{tabular}{|c|c|c|c|}
            \hline
            \textbf{instance} & \textbf{\#jader} & \textbf{\#node} & \multicolumn{1}{c|}{\textbf{doba běhu} {[}m{]}} \\ \hline
            \multirow{10}{*}{saj7}  & 6  & 3 & 1.891 \\
            & 6  & 4 & 1.630 \\
            & 8  & 3 & 1.518 \\
            & 8  & 4 & 1.318 \\
            & 12 & 3 & 1.316 \\
            & 12 & 4 & 1.210 \\
            & 16 & 3 & 1.294 \\
            & 16 & 4 & 1.141 \\
            & 20 & 3 & 1.269 \\
            & 20 & 4 & 1.210 \\ \hline
            \multirow{10}{*}{saj10} & 6  & 3 & 0.912 \\
            & 6  & 4 & 0.583 \\
            & 8  & 3 & 0.787 \\
            & 8  & 4 & 0.542 \\
            & 12 & 3 & 0.789 \\
            & 12 & 4 & 0.502 \\
            & 16 & 3 & 0.713 \\
            & 16 & 4 & 0.520 \\
            & 20 & 3 & 0.746 \\
            & 20 & 4 & 0.521 \\ \hline
            \multirow{10}{*}{saj12} & 6  & 3 & 4.111 \\
            & 6  & 4 & 3.578 \\
            & 8  & 3 & 3.409 \\
            & 8  & 4 & 3.054 \\
            & 12 & 3 & 3.267 \\
            & 12 & 4 & 2.638 \\
            & 16 & 3 & 3.021 \\
            & 16 & 4 & 2.783 \\
            & 20 & 3 & 2.703 \\
            & 20 & 4 & 2.849 \\ \hline
        \end{tabular}
        \caption{MPI algoritmus.}
        \label{tab:mpi}
    \end{table}



    \section{Závěr}

%    Celkove zhodnoceni semestralni prace a zkusenosti ziskanych behem semestru.
    Cílem předmětu bylo si na jednoduchém úkolu prohledávání stavového prostoru v šachovnici vyzkoušet metody pro
    nalezení optimálního řešení. Nejdříve jsem implementoval jednoduché sekveční řešení. To jsem dále paralelizoval
    s použitím OpenMP. Přitom jsem se naučil jak s OpenMP zacházet a dva způsoby paralelizace – datová a tasková.
    Použití OpenMP mi nedělalo větší problémy, protože se stačí pouze zamyslet a chytře do sekvenčného kódu doplnit
    pár direktiv případně dopsat jednu/dvě funkce. Navíc se pod OpenMP dá kód rozumně debugovat. Větší problém jsem
    měl s MPI. Pro jeho použití jsem musel doplnit a přepsat značnou část fungujícího OpenMP kódu. Nejvíce práce na
    MPI mi zabrala serializace/deserializace instancí a řešení deadlocků při posílání.

    \newpage
    \section{Spuštění a překlad}
    Skripty ve výpisech \ref{listing:openmp-runner} a \ref{listing:mpi-runner}  automatizují překlad a spuštění OpenMP a MPI řešení na svazku STAR.
    Pro sekvenční úlohu jsem skript nevytvářel, protože rychlejší bylo ruční měření a překlad. Kompletní
    skripty i s konfiguračními soubory jsou volně dostupná na \url{https://github.com/TaIos/ni-pdp-vysledky-mereni-star}.
    Pro stručnost uvádím tabulku \ref{tab:compilation-short}, která zobrazuje pouze příkazy, které skripty používají pro překlad.

    \begin{table}[ht]
        \centering
        \begin{tabular}{|l|l|}
            \hline
            \textbf{řešení}    & \textbf{příkaz}                                      \\ \hline
            sekvenční & \const{g++ --std=c++11 -O3 -funroll-loops}          \\ \hline
            OpenMP    & \const{g++ --std=c++11 -O3 -funroll-loops -fopenmp} \\ \hline
            MPI       & \const{mpicxx --std=c++11 -lm -O3 -funroll-loops -fopenmp}   \\ \hline
        \end{tabular}
        \caption{Kompilační příkazy používané skripty.}
        \label{tab:compilation-short}
    \end{table}



    \pagebreak
    \newgeometry{top=20pt}
    \begin{figure}[]
        \centering
        \begin{cminted}[fontsize=\small]{bash}
CPP_PROGRAM_TEMPLATE='main.template.cpp'
RUN_SCRIPT_TEMPLATE='../serial_job.template.sh'
CPP_COMPILE="g++"
CPP_FLAGS="--std=c++11 -O3 -funroll-loops -fopenmp"
QRUN_CMD="qrun2 20c 1 pdp_serial"
DATA_PATH="/home/saframa6/ni-pdp-semestralka/data"

createDirectory() {
    if [ ! -d ${1} ]
        then
        mkdir -p ${1}
    fi
}

cd ${1:-$(pwd)}
OUT_DIR="out$(find . -mindepth 1 -maxdepth 1 | sed 's/^\.\///g' | grep -P '^out\d*$' | wc -l)"

createDirectory ${OUT_DIR}

INSTANCES=(7 10 12) # saj instance id
PROCNUMS=(1 2 4 6 8 10 16 20) # number of openmp cores
for INSTANCE in ${INSTANCES[*]}
do
    for PROCNUM in ${PROCNUMS[*]}
    do
        WORKDIR=$(realpath "${OUT_DIR}/saj${INSTANCE}-p${PROCNUM}")
        createDirectory ${WORKDIR}

        CPP_PROGRAM=$(realpath "${WORKDIR}/main.cpp")
        EXE_PROGRAM=$(realpath "${WORKDIR}/run.out")
        RUN_SCRIPT=$(realpath "${WORKDIR}/openmp-job-saj${INSTANCE}-p${PROCNUM}.sh")
        STDERR=$(realpath ${WORKDIR}/stderr)
        STDOUT=$(realpath ${WORKDIR}/stdout)
        touch ${STDERR} ${STDOUT}

        echo $WORKDIR
        echo -e "\tCPP program: ${CPP_PROGRAM}"
        echo -e "\tEXE program: ${EXE_PROGRAM}"
        echo -e "\tRUN script: ${RUN_SCRIPT}"

        sed "s/{PROCNUM}/$PROCNUM/g" ${CPP_PROGRAM_TEMPLATE} > ${CPP_PROGRAM}
        echo -e "\tCOMPILE: ${CPP_COMPILE} ${CPP_FLAGS} ${CPP_PROGRAM} -o ${EXE_PROGRAM}"
        ${CPP_COMPILE} ${CPP_FLAGS} ${CPP_PROGRAM} -o ${EXE_PROGRAM}

        sed "
            s|{EXE_PROGRAM}|$EXE_PROGRAM|g;
            s|{ARGUMENTS}|$DATA_PATH/saj$INSTANCE.txt|g;
            s|{STDOUT}|$STDOUT|g;
            s|{STDERR}|$STDERR|g;
            " ${RUN_SCRIPT_TEMPLATE} > ${RUN_SCRIPT}
        echo -e "\tQRUN: ${QRUN_CMD} ${RUN_SCRIPT}"

        ${QRUN_CMD} ${RUN_SCRIPT}
        echo "============================="
    done
done

echo "DONE"
exit 0
        \end{cminted}
        \caption{Bash skript \href{https://github.com/TaIos/ni-pdp-vysledky-mereni-star/blob/master/openmp/tester-openmp.sh}{tester-openmp.sh} pro spuštění a otestování OpenMP
                    řešení na svazku STAR.}
        \label{listing:openmp-runner}
    \end{figure}

\pagebreak

\begin{figure}[]
    \centering
    \begin{cminted}[fontsize=\small]{bash}
CPP_PROGRAM_TEMPLATE='main.template.cpp'
RUN_SCRIPT_TEMPLATE='parallel_job.template.sh'
CPP_COMPILE="mpicxx"
CPP_FLAGS="--std=c++11 -lm -O3 -funroll-loops -fopenmp"
QRUN_CMD_TEMPLATE="qrun2 20c {NODENUM} pdp_long"  # pdp_fast/pdp_long
DATA_PATH="/home/saframa6/ni-pdp-semestralka/data"

createDirectory() {
    if [ ! -d ${1} ]
        then
        mkdir -p ${1}
    fi
}

cd ${1:-$(pwd)}
OUT_DIR="out$(find . -mindepth 1 -maxdepth 1 | sed 's/^\.\///g' | grep -P '^out\d*$' | wc -l)"

createDirectory ${OUT_DIR}

INSTANCES=(7 10 12) # saj instance id
PROCNUMS=(6 8 12 16 20) # number of openmp cores
NODENUMS=(3 4) # total number of MPI nodes
for INSTANCE in ${INSTANCES[*]}
do
    for PROCNUM in ${PROCNUMS[*]}
    do
        for NODENUM in ${NODENUMS[*]}
        do
            WORKDIR=$(realpath "${OUT_DIR}/saj${INSTANCE}-n${NODENUM}-p${PROCNUM}")
            createDirectory ${WORKDIR}

            CPP_PROGRAM=$(realpath "${WORKDIR}/main.cpp")
            EXE_PROGRAM=$(realpath "${WORKDIR}/run.out")
            RUN_SCRIPT=$(realpath "${WORKDIR}/mpi-job-saj${INSTANCE}-n${NODENUM}-p${PROCNUM}.sh")
            STDERR=$(realpath ${WORKDIR}/stderr)
            STDOUT=$(realpath ${WORKDIR}/stdout)
            touch ${STDERR} ${STDOUT}

            echo $WORKDIR
            echo -e "\tCPP program: ${CPP_PROGRAM}"
            echo -e "\tEXE program: ${EXE_PROGRAM}"
            echo -e "\tRUN script: ${RUN_SCRIPT}"

            QRUN_CMD=$(sed "s/{NODENUM}/${NODENUM}/g"  <<< ${QRUN_CMD_TEMPLATE})

            sed "s/{PROCNUM}/$PROCNUM/g" ${CPP_PROGRAM_TEMPLATE} > ${CPP_PROGRAM}

            echo -e "\tCOMPILE: ${CPP_COMPILE} ${CPP_FLAGS} ${CPP_PROGRAM} -o ${EXE_PROGRAM}"
            ${CPP_COMPILE} ${CPP_FLAGS} ${CPP_PROGRAM} -o ${EXE_PROGRAM}

            sed "
                s|{EXE_PROGRAM}|$EXE_PROGRAM|g;
                s|{ARGUMENTS}|$DATA_PATH/saj$INSTANCE.txt|g;
                s|{STDOUT}|$STDOUT|g;
                s|{STDERR}|$STDERR|g;
                " ${RUN_SCRIPT_TEMPLATE} > ${RUN_SCRIPT}
            echo -e "\tQRUN: ${QRUN_CMD} ${RUN_SCRIPT}"

            ${QRUN_CMD} ${RUN_SCRIPT}
            echo "============================="
        done
    done
done

echo "DONE"
exit 0
    \end{cminted}
    \caption{Bash skript \href{https://github.com/TaIos/ni-pdp-vysledky-mereni-star/blob/master/mpi/tester-mpi.sh}{tester-mpi.sh} pro spuštění a otestování MPI
                    řešení na svazku STAR. }
    \label{listing:mpi-runner}
\end{figure}

\restoregeometry

\section{Literatura}
    Tenderloin pork belly ham leberkas doner rump. Filet mignon beef pastrami pork belly drumstick. Beef ribs filet mignon porchetta pork turducken spare ribs tri-tip corned beef strip steak turkey capicola. Venison hamburger ball tip, buffalo fatback pork alcatra doner pork belly.


\end{document}
