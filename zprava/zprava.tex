\documentclass{article}
\usepackage{minted}
\usepackage{cprotect}
\oddsidemargin=-5mm
\evensidemargin=-5mm\marginparwidth=.08in \marginparsep=.01in
\marginparpush=5pt\topmargin=-15mm\headheight=12pt
\headsep=25pt
%\footheight=12pt
\footskip=30pt\textheight=25cm
\textwidth=17cm\columnsep=2mm
\columnseprule=1pt\parindent=15pt\parskip=2pt
\renewcommand{\figurename}{Obrázek}
\usepackage{fancyvrb}
\usepackage{listings}


\newcommand{\heuristika}[1]{\textbf{\textit{Heuristika {#1}.}}}

%%%%%%%%%%%%%%%%%%%%%%%%%%%%%%%%%%%%%%%%%%%%%%%%%%%%%%%%%%%%%%%%%%%%%%%%%%%%%%%
%http://ftp.cvut.cz/tex-archive/macros/latex/contrib/algorithm2e/doc/algorithm2e.pdf
%\usepackage[ruled,vlined]{algorithm2e}
\usepackage[linesnumbered,lined,boxed,commentsnumbered]{algorithm2e}
%%%%%%%%%%%%%%%%%%%%%%%%%%%%%%%%%%%%%%%%%%%%%%%%%%%%%%%%%%%%%%%%%%%%%%%%%%%%%%%


\begin{document}
    \begin{center}
        \bf Semestralni projekt MI-PDP 2020/2021:\\[5mm]
        Paralelni algoritmus pro reseni problemu\\[5mm]
        Martin Šafránek\\[2mm]
        magisterske studijum, FIT CVUT, Thakurova 9, 160 00 Praha 6\\[2mm]
        \today
    \end{center}


    \section{Definice problemu a popis sekvencniho algoritmu}

    Program řeší problém nalezení optimální posloupnosti tahů pro střelce a jezdce, která vede k sebrání
    všech pěšců rozmístěných na šachovnici. Jedná se o analogii problému obchodního cestujícího.
    Nalezení optimálního řešení je proto NP těžký úkol. Řešení v této práci používá bruteforce
    s heuristikami pro ořezávání stavového prostoru.

    Příklad vstupu je na obrázku \ref{fig:board-example}. Obsahuje vždy
    \begin{enumerate}
        \item přirozené číslo $k$, reprezentující délku strany šachovnice $S$ o velikosti $k \times k$,
        \item pole souřadnic rozmístěných figurek na šachovnici S,
        \item horní mez délky optimální posloupnosti $d^{*}_{max}$.
    \end{enumerate}

    \begin{figure}[h]
        \begin{center}
            \begin{BVerbatim}
                11
                22
                -----------
                -------P---
                P-----P--P-
                ---PJS--P--
                ----P------
                ---P--P----
                ------P----
                ------P----
                ------P----
                -----PP----
                -----------
            \end{BVerbatim}
        \end{center}
        \cprotect\caption{ Příklad vstupních dat pro $k=11$, $d^{*}_{max}=22$. Střelec je označen \verb|S|, jezdec \verb|J|,\
            pěšák \verb|P| a prázdné políčko \verb|-|. }
        \label{fig:board-example}
    \end{figure}

    Sekvenční algoritmus je popsán v \ref{alg:sequential}. Používá dvě heuristiky.

    \heuristika{střelec} Z množiny možných políček, kam je možné střelce přemístit jsou preferována ty,
    která obsahují pěšce. Pokud takové políčko neexistuje, jsou preferována políčka s alespoň jedním pěšákem
    na diagonále. Jinak se pohyb střelce rozhodne náhodně.
    \hspace{1pt}


    \heuristika{kůň} Z množiny možných políček, kam je možné koně přemístit jsou preferována ty,
    která obsahují pěšce. Pokud takové políčko neexistuje, jsou preferována políčka, z kterých
    kůň ve svém následujícím tahu může vzít pěšáka. Pokud ani takové políčko neexistuje, jsou
    preferováno políčka, z kterých kůň v následujícíh dvou tazích může vzít pěšáka.
    Jinak se pohyb koně rozhodne náhodně.

    \begin{algorithm}[H]
        \SetKwInOut{Input}{input}\SetKwInOut{Output}{output}
        \SetAlgorithmName{Algoritmus}
        \SetAlgoLined

        \Input{$k\times k$ pole, mez $d^{*}_{max}$}
        \Output{optimální posloupnost tahů}
        \BlankLine

        \eIf{abc or def}{

        }\;
        \While{While condition}{
            instructions\;
            \eIf{condition}{
                instructions1\;
                instructions2\;
            }{
                instructions3\;
            }
        }
        \caption{sekvenční}
        \label{alg:sequential}
    \end{algorithm}

    \section{Popis paralelniho algoritmu a jeho implementace v OpenMP - taskovy paralelismus}

    Popiste paralelni algoritmus, opet vyjdete ze zadani a presne vymezte
    odchylky, ktere pri implementaci OpenMP pouzivate.
    Popiste a vysvetlete strukturu celkoveho
    paralelniho algoritmu na urovni procesuu v OpenMP a strukturu kodu
    jednotlivych procesu. Napr. jak je naimplemtovana smycka pro cinnost
    procesu v aktivnim stavu i v stavu necinnosti. Jake jste zvolili
    konstanty a parametry pro skalovani algoritmu. Struktura a semantika
    prikazove radky pro spousteni programu.


    \section{Popis paralelniho algoritmu a jeho implementace v OpenMP - datovy paralelismus}

    Popiste paralelni algoritmus, opet vyjdete ze zadani a presne vymezte
    odchylky, ktere pri implementaci OpenMP pouzivate.
    Popiste a vysvetlete strukturu celkoveho
    paralelniho algoritmu na urovni procesuu v OpenMP a strukturu kodu
    jednotlivych procesu. Napr. jak je naimplemtovana smycka pro cinnost
    procesu v aktivnim stavu i v stavu necinnosti. Jake jste zvolili
    konstanty a parametry pro skalovani algoritmu. Struktura a semantika
    prikazove radky pro spousteni programu.


    \section{Popis paralelniho algoritmu a jeho implementace v MPI}

    Popiste paralelni algoritmus, opet vyjdete ze zadani a presne vymezte
    odchylky, zvlaste u Master-Slave casti. Popiste a vysvetlete strukturu celkoveho
    paralelniho algoritmu na urovni procesuu v MPI a strukturu kodu
    jednotlivych procesu. Napr. jak je naimplemtovana smycka pro cinnost
    procesu v aktivnim stavu i v stavu necinnosti. Jake jste zvolili
    konstanty a parametry pro skalovani algoritmu. Struktura a semantika
    prikazove radky pro spousteni programu.


    \section{Namerene vysledky a vyhodnoceni}

    \begin{enumerate}
        \item Zvolte tri instance problemu s takovou velikosti vstupnich dat, pro ktere ma sekvencni
        algoritmus casovou slozitost mezi 1 a 10 minutami.
        Pro mereni cas potrebny na cteni dat z disku a ulozeni na disk neuvazujte a zakomentujte
        ladici tisky, logy, zpravy a vystupy.
        \item Merte paralelni cas pri pouziti $i=2,\cdot,60$ vypocetnich jader.
        \item Tabulkova a pripadne graficky zpracovane namerene hodnoty casove slozitosti měernych instanci behu programu s popisem instanci dat. Z namerenych dat sestavte grafy zrychleni $S(n,p)$.
        \item Analyza a hodnoceni vlastnosti paralelniho programu, zvlaste jeho efektivnosti a skalovatelnosti, pripadne popis zjisteneho superlinearniho zrychleni.

    \end{enumerate}


    \section{Zaver}

    Celkove zhodnoceni semestralni prace a zkusenosti ziskanych behem semestru.


    \section{Literatura}


\end{document}
