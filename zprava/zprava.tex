\documentclass{article}
\usepackage{minted}
\usepackage{cprotect}
\usepackage{hyperref}
\usepackage{geometry}
\oddsidemargin=-5mm
\evensidemargin=-5mm\marginparwidth=.08in \marginparsep=.01in
\marginparpush=5pt\topmargin=-15mm\headheight=12pt
\headsep=25pt
%\footheight=12pt
\footskip=30pt\textheight=25cm
\textwidth=17cm\columnsep=2mm
\columnseprule=1pt\parindent=15pt\parskip=2pt
\renewcommand{\figurename}{Obrázek}
\usepackage{fancyvrb}
\usepackage{listings}
\usepackage[czech]{babel}
\newcommand{\heuristika}[1]{\textbf{\textit{Heuristika {#1}.}}}
\newcommand{\const}[1]{\mintinline{bash}{#1}}
\newcommand{\konstanta}[1]{\textbf{\textit{Konstanta}} \const{#1}.}

% minted centering
\usepackage{xpatch,letltxmacro}
\LetLtxMacro{\cminted}{\minted}
\let\endcminted\endminted
\xpretocmd{\cminted}{\RecustomVerbatimEnvironment{Verbatim}{BVerbatim}{}}{}{}

%%%%%%%%%%%%%%%%%%%%%%%%%%%%%%%%%%%%%%%%%%%%%%%%%%%%%%%%%%%%%%%%%%%%%%%%%%%%%%%
%http://ftp.cvut.cz/tex-archive/macros/latex/contrib/algorithm2e/doc/algorithm2e.pdf
%\usepackage[ruled,vlined]{algorithm2e}
\usepackage[linesnumbered,lined,boxed,commentsnumbered]{algorithm2e}
%%%%%%%%%%%%%%%%%%%%%%%%%%%%%%%%%%%%%%%%%%%%%%%%%%%%%%%%%%%%%%%%%%%%%%%%%%%%%%%

\usepackage{multirow}
\usepackage{subcaption}

\begin{document}
    \begin{center}
        \bf Semestrální projekt MI-PDP 2020/2021:\\[5mm]
        Paralelní algoritmus pro řesení problému prohledávání stavového prostoru\\[5mm]
        Martin Šafránek\\[2mm]
        magisterské studijum, FIT ČVUT, Thákurova 9, 160 00 Praha 6\\[2mm]
        zdrojové kódy:\\ \url{https://github.com/TaIos/ni-pdp-semestralka}\\[2mm]
        výsledky měření:\\ \url{https://github.com/TaIos/ni-pdp-vysledky-mereni-star}\\[2mm]
        \today
    \end{center}


    \section{Definice problému a popis sekvenčního algoritmu}
    \label{sec:sequential1}

    Program řeší problém nalezení optimální posloupnosti tahů pro střídavé tahy střelce a jezdce, která vede k sebrání
    všech pěšců rozmístěných na šachovnici. První tah je proveden střelcem. Jedná se o analogii problému obchodního cestujícího.
    Nalezení optimálního řešení je proto NP těžký úkol. Řešení v této práci používá bruteforce
    s heuristikami pro ořezávání stavového prostoru.
    \newline

    \label{subsec:popis-vstupu-vystupu}
    \subsection{Popis vstupu a výstupu}
    Příklad vstupu je uveden na obrázku \ref{fig:board-example}. Obsahuje popořadě vždy
    \begin{enumerate}
        \item přirozené číslo $k$, reprezentující délku strany šachovnice $S$ o velikosti $k \times k$,
        \item horní mez délky optimální posloupnosti $d^{*}_{max}$,
        \item pole souřadnic rozmístěných figurek na šachovnici S.
    \end{enumerate}
    Výstup obsahuje optimální posloupnost tahů pro sebrání všech pěšců na šachovnici. Příklad výsputu je na obrázku  \ref{fig:output-example}.

    \begin{figure}[h]
        \begin{center}
            \begin{BVerbatim}
                11
                22
                -----------
                -------P---
                P-----P--P-
                ---PJS--P--
                ----P------
                ---P--P----
                ------P----
                ------P----
                ------P----
                -----PP----
                -----------
            \end{BVerbatim}
        \end{center}
        \cprotect\caption{ Příklad vstupních dat pro $k=11$, $d^{*}_{max}=22$. Střelec je označen \verb|S|, jezdec \verb|J|,\
            pěšák \verb|P| a prázdné políčko \verb|-|. }
        \label{fig:board-example}
    \end{figure}

    \begin{figure}[h]
        \begin{center}
            \begin{BVerbatim}
                4,4 *
                2,1
                3,3
                1,3 *
                4,2 *
            \end{BVerbatim}
        \end{center}
        \caption{Příklad výstupu. Každá řádka obsahuje souřadnice na šachovnici, kam se v daném tahu kůň nebo střelec přesunul. Hvězdička značí, že při daném tahu byl sebrán pěšec.}
        \label{fig:output-example}
    \end{figure}


    \pagebreak
    \label{subsec:seq-heuristics}


    \subsection{Heuristiky}
    Sekvenční algoritmus používá dvě heuristiky pro pohyb střelce a koně.

    \heuristika{střelec} Z množiny možných políček, kam je možné střelce přemístit jsou preferována ty,
    která obsahují pěšce. Pokud takové políčko neexistuje, jsou preferována políčka s alespoň jedním pěšákem
    na diagonále. Jinak se pohyb střelce rozhodne náhodně.
    \hspace{1.5pt}

    \heuristika{kůň} Z množiny možných políček, kam je možné koně přemístit jsou preferována ty,
    která obsahují pěšce. Pokud takové políčko neexistuje, jsou preferována políčka, z kterých
    kůň ve svém následujícím tahu může vzít pěšáka. Pokud ani takové políčko neexistuje, jsou
    preferována políčka, z kterých kůň v následujícíh dvou tazích může vzít pěšáka.
    Jinak se pohyb koně rozhodne náhodně.

    \subsection{Naměřené sekvenčí časy}

    \begin{table}[hb!]
        \centering
        \begin{tabular}{|c|c|}
            \hline
            \multicolumn{1}{|l|}{\textbf{instance}} & \multicolumn{1}{l|}{\textbf{{}doba běhu} {[}m{]}} \\ \hline
            saj7                           & 1.727                                       \\ \hline
            saj10                          & 8.163                                       \\ \hline
            saj12                          & 4.542                                       \\ \hline
        \end{tabular}
        \caption{Sekvenční algoritmus -- doba běhu v minutách.}
        \label{tab:seq}
    \end{table}

    \subsection{Pseudokód sekvenčního algoritmu}
    \begin{algorithm}[H]
        \SetAlgoLined
        \DontPrintSemicolon
        \SetAlgorithmName{Algoritmus}{}{}
        \SetKwFunction{betterSolutionExists}{existujeLepšíŘešení}
        \SetKwFunction{updateBestSolution}{aktualizujNejlepšíŘešení}
        \SetKwFunction{HeuristikaKun}{HeuristikaKůň}
        \SetKwFunction{HeuristikaStrelec}{HeuristikaStřelec}
        \SetKwFunction{MakeMove}{ProveďTah}
        \SetKwFunction{FBetterSolutionExists}{existujeLepšíŘešení}

        \SetKwProg{Fn}{Function}{:}{}
        \Fn{\FBetterSolutionExists{šachovnice $S$, nejlepší řešení $S^*$}}{
        \uIf{počet pěšáků $S$ + počet tahů $S$ $\ge$ délka $S^*$
            \newline {\bf or}
            počet tahů $S$ + počet pěšáků $S$ > maximální hloubka
            \newline {\bf or}
            $S^*$ má minimální možný počet tahů
        } {
                \Return {\bf True}\;
            } \Else{
            \Return {\bf False}\;
        }\;
        }\;


        \SetKwFunction{FMain}{sequence}
        \SetKwProg{Fn}{Function}{:}{}
        \Fn{\FMain{šachovnice $S$, nejlepší řešení $S^*$}}{
            \If{\betterSolutionExists{S, S*}} {
                \Return
            }\;
            \If{počet pěšáků $S$ je 0} {
                \updateBestSolution{S, S*}\;
                \Return\;
            }\;

            \SetKwData{Tahy}{Tahy}
            \Tahy$\leftarrow$ prázdný list\;
        \If{je na tahu kůň}{
            \Tahy = \HeuristikaKun{$S$}
        }\;
            \If{je na tahu střelec}{
                \Tahy = \HeuristikaStrelec{$S$}
            }\;

            \SetKwData{Next}{$S_{next}$}
            \ForAll{$t$ in \Tahy} {
                \Next = \MakeMove{$S$, $t$}\;
                \FMain{\Next, $S^*$}\;
            }\;
        }\;
        \caption{sekvenční}
        \label{alg:sequential}
    \end{algorithm}

    \label{sec:task-par}


    \section{Popis paralelního algoritmu a jeho implementace v OpenMP - taskový paralelismus}

    Taskový paralelní algoritmus je naimplementován pomocí OpenMP. Hlavní rozdíl oproti sekvenčnímu algoritmu
    popsaném v sekci \ref{sec:sequential1} je rozdělení úlohy prohledávání stavového prostoru na tasky. Task je
    základní jednotka, kterou je OpenMP schopno přidělit vláknu a provést tak výpočet. Pro zadanou úlohu
    task znamená šachovnici s pozicí všech figurek a historií tahů. Takto vytvořené tasky OpenMP přidává
    do svého taskpoolu, z kterého si je vlákna vyzvedávají a řeší. Dále všechny vlákna řešící tasky z
    taskpoolu sdílejí nejlepší řešení $d_{best}$. Heuristiky jsou totožné jako v podsekci \ref{subsec:seq-heuristics}.

    \subsection{Konstanty a parametry pro škálování algoritmu}
    Taskový paralelní algoritmus implementovaný pomocí OpenMP umožňuje nastavení konstant, které ovlivní
    logiku funkce programu a tedy i výpočetní čas. Změněny byly pouze zde zmíněné konstanty. Jejich hodnota
    byla určena empiricky na vstupních datech pomocí měření. Nejedná se o optimální hodnoty, protože jejich nalezení je
    stejně těžký problém jako nalezení optimální cesty v původním problému.

    \pagebreak
    \konstanta{TASK_THRESHOLD}  Pokud vlákno řeší instanci a délka její cesty je delší než \const{TASK_THRESHOLD}, nevytváří další
    OpenMP tasky a nepřidává je do taskpoolu. Zadanou instaci vyřeší použitím sekvenčního algoritmu popsaném v sekci \ref{sec:sequential1}.

    \begin{table}[hb]
        \centering
        \begin{tabular}{|l|l|}
            \hline
            název                  & hodnota \\ \hline
            \const{TASK_THRESHOLD} & 4       \\ \hline
        \end{tabular}
        \caption{Konstanty použité v OpenMP taskovém paralelismu.}
        \label{tab:data-par-constants}
    \end{table}

    \subsection{Pseudokód}

    \begin{algorithm}[H]
        \SetAlgoLined
        \DontPrintSemicolon
        \SetAlgorithmName{Algoritmus}{}{}
        \SetKwFunction{betterSolutionExists}{existujeLepšíŘešení}
        \SetKwFunction{UpdateBestSolution}{aktualizujNejlepšíŘešení}
        \SetKwFunction{HeuristikaKun}{HeuristikaKůň}
        \SetKwFunction{HeuristikaStrelec}{HeuristikaStřelec}
        \SetKwFunction{MakeMove}{ProveďTah}
        \SetKwFunction{FBetterSolutionExists}{existujeLepšíŘešení}
        \SetKwFunction{Threshold}{TASK\_THRESHOLD}

        \SetKwFunction{FMain}{openMpTask}
        \SetKwProg{Fn}{Function}{:}{}
        \Fn{\FMain{šachovnice $S$, nejlepší řešení $S^*$}}{
            \If{\betterSolutionExists{S, S*}}{
                \Return\;
            }\;

        \If{počet pěšáků $S$ je 0} {
            \If{\betterSolutionExists{S, S*}}{
                \texttt{\#pragma omp critical}\;
                \If{\betterSolutionExists{S, S*}}{
                    \UpdateBestSolution{S, S*}\;
                }\;
            }\;
            \Return\;
        }\;

            \SetKwData{Tahy}{Tahy}
            \Tahy$\leftarrow$ prázdný list\;
            \If{je na tahu kůň}{
                \Tahy = \HeuristikaKun{$S$}
            }\;
            \If{je na tahu střelec}{
                \Tahy = \HeuristikaStrelec{$S$}
            }\;

            \SetKwData{Next}{$S_{next}$}
            \ForAll{$t$ in \Tahy} {
                \Next = \MakeMove{$S$, $t$}\;
                \uIf{počet tahů S > \Threshold}{
                    \FMain{\Next, $S^*$}\;
                } \Else{
                    \texttt{\#pragma omp task firstprivate($\ldots$)}\;
                    \FMain{\Next, $S^*$}\;
                }\;
            }\;
        }\;
        \caption{OpenMP task}
        \label{alg:openmp-task}
    \end{algorithm}


    \pagebreak
    \label{sec:data-par}
    \section{Popis paralelniho algoritmu a jeho implementace v OpenMP - datový paralelismus}
    Datový paralelismus v OpenMP pracuje s datově nezávislými celky, které podle určené
    strategie přiděluje vláknům na zpracování. Nezávislý datový celek je pro zadanou úlohu
    šachovnice s pozicí všech figurek a historií tahů.

    První krok je vygenerování datově nezávislých celků – to je provedeno před použitím OpenMP.
    Ty jsou následně najednou předány OpenMP. To je určenou strategií rozdělí mezi vlákna.
    Každé vlákno pak provádí sekvenční řešení problému popsané v sekci \ref{sec:sequential1}.
    Vlákna mezi sebou sdílejí pouze nejlepší řešení.

    \subsection{Konstanty a parametry pro škálování algoritmu}
    Prvním krokem před spuštěním OpenMP řešení je vytvoření datově nezávislých instancí.
    Ty se vytvoří použitím mírně upraveného sekvečního algoritmu \ref{alg:sequential}.
    Jejich počet je regulován konstantou \const{EPOCH_CNT}.

    Parametrem OpenMP je konstanta \const{schedule}. Ta určuje politiku přidělování datově nezávislých instací vláknům.
    Zde je použitá hodnota \const{dynamic()} bez parametrů. To znamená, že pokud vlákno dokončí výpočet je mu
    přiřazena jedna další datově nezávislá instance k vyřešení.

    \konstanta{EPOCH_CNT} Určuje, kolik datově nezávislých instancí je vygenerováno.
    Pro každou epochu jsou provedeny všechny možné tahy buď koněm, nebo střelcem.
    Po vyčerpání všech epoch jsou stavy, do kterých se kůň a střelec dostali
    použity jako nezávislé instance.

    \konstanta{schedule} OpenMP konstanta, která nastavuje politiku přidělování datově nezávislých instací vláknům.

    \begin{table}[hb]
        \centering
        \begin{tabular}{|l|l|}
            \hline
            název      & hodnota \\ \hline
            \const{EPOCH_CNT} & 3       \\ \hline
            \const{schedule} & dynamic()      \\ \hline
        \end{tabular}
        \caption{Konstanty použité v OpenMP datovém paralelismu.}
        \label{tab:data-par-constants}
    \end{table}

    \subsection{Pseudokód paralelního algoritmu — datový paralelismus}
    Datový paralelismus použává sekvenční algoritmus \ref{alg:sequential}.
    Přidává navíc synchronizaci mezi vlákny pro přepisování nejlepšího řešení $S^*$.
    Ta je identická jako u taskového OpenMP algoritmu \ref{alg:openmp-task}.

    \subsection{Pseudokód}

    \begin{algorithm}[H]
        \SetAlgoLined
        \DontPrintSemicolon
        \SetAlgorithmName{Algoritmus}{}{}
        \SetKwFunction{GenerateInstances}{vygenerujInstance}
        \SetKwFunction{SequenceWithOmpCritical}{sequenceAlgorgitmusSOmpCriticalProUpdate}
        \SetKwData{Instance}{Instance}

        \SetKwFunction{FMain}{openMpData}
        \SetKwProg{Fn}{Function}{:}{}
        \Fn{\FMain{šachovnice $S$, nejlepší řešení $S^*$}}{
            \Instance$\leftarrow$ \GenerateInstances{$S$}\;
            \texttt{\#pragma omp parallel for $\ldots$}\;
            \ForAll{$S_{gen}$ in \Instance} {
            \SequenceWithOmpCritical{$S_{gen}$, $S^*$}
            }\;
        }\;
        \caption{OpenMP data}
        \label{alg:openmp-data}
    \end{algorithm}


    \pagebreak
    \label{sec:mpi}
    \section{Popis paralelního algoritmu a jeho implementace v MPI}

    Řešení s použitím MPI se skládá ze dvou částí. První je datový OpenMP paralelismus,
    viz sekce \ref{sec:data-par}. Druhou část tvoří MPI. To má za úkol řídit a distribuovat
    výpočet na několika výpočetních uzlech.

    MPI řešení začíná tím, že si master proces identickým způsobem jako v sekci s datovým paralelismem
    vygeneruje datově nezávislé instance. Ty pak serializuje a společně s globálním nejlepším řešením je
    pošle přes MPI interface slave procesům. Každý z nich pomocí datového paralelismu vyřeší přijmuté řešení a
    odešlě ho zpět master procesu. Pak požádá master proces o další instanci k vyřešení.
    Pokud master vláknu nemá už žádné instance k vyřešení, pošle slave procesu ukončující zprávu.
    Více viz algoritmus \ref{alg:mpi}.

    \subsection{Konstanty a parametry pro škálování algoritmu}
    Použité konstanty jsou shodné jako v sekci \ref{sec:data-par}, protože MPI využívá pro řešení instancí
    OpenMP datový paralelismus. Pro MPI nebylo potřeba nastavovat žádné konstanty, které ovlivní rychlost
    nalezení řešení.

    \subsection{Pseudokód}

    \begin{algorithm}[H]
        \SetAlgoLined
        \DontPrintSemicolon
        \SetAlgorithmName{Algoritmus}{}{}
        \SetKwFunction{GenerateInstances}{vygenerujInstance}
        \SetKwFunction{SequenceWithOmpCritical}{sequenceAlgorgitmusSOmpCriticalProUpdate}
        \SetKwFunction{MpiSend}{MPI\_Send}
        \SetKwFunction{MpiIProbe}{MPI\_Iprobe($\ldots$)}
        \SetKwFunction{MpiRecv}{MPI\_Recv($\ldots$)}
        \SetKwData{Instance}{Instance}
        \SetKwData{InstanceAti}{Instance[i]}
        \SetKwFunction{UpdateBestSolution}{aktualizujNejlepšíŘešení}
        \SetKwFunction{betterSolutionExists}{existujeLepšíŘešení}
        \SetKwFunction{OpenMpData}{openMpData}


        \SetKwFunction{FMain}{mainMPI}
        \SetKwProg{Fn}{Function}{:}{}
        \Fn{\FMain{šachovnice $S$, nejlepší řešení $S^*$}}{
            \uIf{master vlákno} {
                \Instance$\leftarrow$ \GenerateInstances{$S$}\;
                \For{$i \leftarrow 1 \ldots$ počet slave procesů} {
                \MpiSend{$slave_i$, \InstanceAti, $S^*$}\;
                }\;

                \SetKwData{Cnt}{alive}
                \Cnt $\leftarrow$ počet slave procesů\;
            \While{True} {
                \If{\MpiIProbe} {
                    $S_{slave}$, $id_{slave} \leftarrow$ \MpiRecv\;
                    \If{not \betterSolutionExists{$S_{slave}$, $S^*$}} {
                        \UpdateBestSolution{$S_{slave}$, $S^*$}\;
                    }\;

                \uIf{zvýbá nějaká nevyřešená instance $S'$ v \Instance}{
                    \MpiSend{$id_{slave}$, $S'$, $S^*$}
                }
                \Else{
                    \MpiSend{$id_{slave}$, ukončující token}\;
                    \Cnt $\leftarrow$ \Cnt $-$ 1

                }\;

            \If {\Cnt $=0$}{
                    {\bf break}\;
            }\;
            }\;

            }\;

            } \ElseIf{slave vlákno} {
                \While{True} {
                    \If{\MpiIProbe} {
                        $S'$, $S^*, token \leftarrow$ \MpiRecv\;
                        \uIf{$token$ =  práce} {
                        $S_{slave} \leftarrow$ \OpenMpData{$S'$, $S^*$} \;
                        \MpiSend{$id_{master}$, $S_{slave}$}\;
                        } \ElseIf{$token$ =  konec} {
                            {\bf break}\;
                        }\;

                    }\;
                }\;

            }\;
        }\;
        \caption{MPI}
        \label{alg:mpi}
    \end{algorithm}


    \pagebreak
    \label{sec:vysledky}
    
    \label{sec:namerene-vysledky}
    \section{Naměřené výsledky a vyhodnocení}

    Všechny tabulky s naměřenými daty jsou z důvodu jejich velikosti v sekci \ref{sec:tabulky}.
    Zde je jejich diskuze a vyhodnocení.


    \begin{enumerate}
        \item Zvolte tri instance problemu s takovou velikosti vstupnich dat, pro ktere ma sekvencni
        algoritmus casovou slozitost mezi 1 a 10 minutami.
        Pro mereni cas potrebny na cteni dat z disku a ulozeni na disk neuvazujte a zakomentujte
        ladici tisky, logy, zpravy a vystupy.
        \item Merte paralelni cas pri pouziti $i=2,\cdot,60$ vypocetnich jader.
        \item Tabulkova a pripadne graficky zpracovane namerene hodnoty casove slozitosti měernych instanci behu programu s popisem instanci dat. Z namerenych dat sestavte grafy zrychleni $S(n,p)$.
        \item Analyza a hodnoceni vlastnosti paralelniho programu, zvlaste jeho efektivnosti a skalovatelnosti, pripadne popis zjisteneho superlinearniho zrychleni.
    \end{enumerate}
    \section{Závěr}

%    Celkove zhodnoceni semestralni prace a zkusenosti ziskanych behem semestru.
    Cílem předmětu bylo si na jednoduchém úkolu prohledávání stavového prostoru v šachovnici vyzkoušet metody pro
    nalezení optimálního řešení. Nejdříve jsem implementoval jednoduché sekveční řešení. To jsem dále paralelizoval
    s použitím OpenMP. Přitom jsem se naučil jak s OpenMP zacházet a dva způsoby paralelizace – datová a tasková.
    Použití OpenMP mi nedělalo větší problémy, protože se stačí pouze zamyslet a chytře do sekvenčného kódu doplnit
    pár direktiv případně dopsat jednu/dvě funkce. Navíc se pod OpenMP dá kód rozumně debugovat. Větší problém jsem
    měl s MPI. Pro jeho použití jsem musel doplnit a přepsat značnou část fungujícího OpenMP kódu. Nejvíce práce na
    MPI mi zabrala serializace/deserializace instancí a řešení deadlocků při posílání.

    \newpage
    \section{Spuštění a překlad}
    Pro automatické testování OpenMP a MPI jsem vytvořil dva skripty. Každý z nich upraví zdrojový kód řešení
    podle potřeby a překompiluje ho. Dále vygeneruje qrun skript, který spustí na svazku STAR. Výsledky uloží na
    předem definované místo. Pro zjednodušení uvádím tabulku \ref{tab:compilation-short}, která ukazuje kompilační příkazy a přepínače
    použité jednotlivými skripty.

    Všechny výsledky testování pro OpenMP implementaci byly vygenerovány jedním spuštěním skriptu
    \href{https://github.com/TaIos/ni-pdp-vysledky-mereni-star/blob/master/openmp/tester-openmp.sh}{tester-openmp.sh}.
    Obdobně pro MPI skriptem \href{https://github.com/TaIos/ni-pdp-vysledky-mereni-star/blob/master/mpi/tester-mpi.sh}{tester-mpi.sh}.
    Pro sekvenční řešení stačilo ruční spouštění.


    \begin{table}[ht]
        \centering
        \begin{tabular}{|l|l|}
            \hline
            \textbf{řešení}    & \textbf{příkaz}                                      \\ \hline
            sekvenční & \const{g++ --std=c++11 -O3 -funroll-loops}          \\ \hline
            OpenMP    & \const{g++ --std=c++11 -O3 -funroll-loops -fopenmp} \\ \hline
            MPI       & \const{mpicxx --std=c++11 -lm -O3 -funroll-loops -fopenmp}   \\ \hline
        \end{tabular}
        \caption{Kompilační příkazy používané automatizačními skripty.}
        \label{tab:compilation-short}
    \end{table}

\section{Literatura}
    Pro vypracování tohoto reportu jsem vycházel z vlastních měření na výpočetním svazku STAR dostupného na adrese \url{star.fit.cvut.cz} a
    materiálů předmětu NI-PDP dostupných na \url{https://courses.fit.cvut.cz/}.

    \section{Příloha: Naměřené výsledky v tabulkách}

    \subsection{Naměřené časy}

    \begin{table}[h]
        \centering
        \begin{tabular}{|c|c|c|}
            \hline
            \multicolumn{1}{|l|}{\textbf{instance}} & \multicolumn{1}{l|}{\textbf{\#jader}} & \textbf{doba běhu} {[}m{]} \\ \hline
            \multirow{8}{*}{saj7}  & 1  & 0.251 \\
            & 2  & 0.311 \\
            & 4  & 0.017 \\
            & 6  & 0.075 \\
            & 8  & 0.047 \\
            & 10 & 0.019 \\
            & 16 & 0.004 \\
            & 20 & 0.006 \\ \hline
            \multirow{8}{*}{saj10} & 1  & 2.199 \\
            & 2  & 3.321 \\
            & 4  & 2.077 \\
            & 6  & 1.695 \\
            & 8  & 1.428 \\
            & 10 & 1.078 \\
            & 16 & 1.439 \\
            & 20 & 1.624 \\ \hline
            \multirow{8}{*}{saj12} & 1  & 9.555 \\
            & 2  & $\infty$   \\
            & 4  & 6.303 \\
            & 6  & 8.426 \\
            & 8  & 8.113 \\
            & 10 & 5.946 \\
            & 16 & $\infty$   \\
            & 20 & 5.654 \\ \hline
        \end{tabular}
        \caption{OpenMP task algoritmus -- doba běhu v minutách.}
        \label{tab:openmp-task}
    \end{table}

    \begin{table}[h]
        \centering
        \begin{tabular}{|c|c|c|}
            \hline
            \multicolumn{1}{|l|}{\textbf{instance}} & \multicolumn{1}{l|}{\textbf{\#jader}} & \textbf{doba běhu} {[}m{]} \\ \hline
            \multirow{8}{*}{saj7}  & 1  & 5.207 \\
            & 2  & 4.935 \\
            & 4  & 2.283 \\
            & 6  & 0.944 \\
            & 8  & 0.775 \\
            & 10 & 0.005 \\
            & 16 & 0.007 \\
            & 20 & 0.004 \\ \hline
            \multirow{8}{*}{saj10} & 1  & 2.032 \\
            & 2  & 2.948 \\
            & 4  & 2.133 \\
            & 6  & 1.817 \\
            & 8  & 1.667 \\
            & 10 & 1.642 \\
            & 16 & 1.686 \\
            & 20 & 1.368 \\ \hline
            \multirow{8}{*}{saj12} & 1  & 9.610 \\
            & 2  & 8.232 \\
            & 4  & 6.927 \\
            & 6  & 5.880 \\
            & 8  & 6.409 \\
            & 10 & 6.695 \\
            & 16 & 4.788 \\
            & 20 & 8.986 \\ \hline
        \end{tabular}
        \caption{OpenMP data algoritmus -- doba běhu v minutách.}
        \label{tab:openmp-data}
    \end{table}

    \begin{table}[h]
        \centering
        \begin{tabular}{|c|c|c|c|}
            \hline
            \textbf{instance} & \textbf{\#jader} & \textbf{\#node} & \multicolumn{1}{c|}{\textbf{doba běhu} {[}m{]}} \\ \hline
            \multirow{10}{*}{saj7}  & 6  & 3 & 1.891 \\
            & 6  & 4 & 1.630 \\
            & 8  & 3 & 1.518 \\
            & 8  & 4 & 1.318 \\
            & 12 & 3 & 1.316 \\
            & 12 & 4 & 1.210 \\
            & 16 & 3 & 1.294 \\
            & 16 & 4 & 1.141 \\
            & 20 & 3 & 1.269 \\
            & 20 & 4 & 1.210 \\ \hline
            \multirow{10}{*}{saj10} & 6  & 3 & 0.912 \\
            & 6  & 4 & 0.583 \\
            & 8  & 3 & 0.787 \\
            & 8  & 4 & 0.542 \\
            & 12 & 3 & 0.789 \\
            & 12 & 4 & 0.502 \\
            & 16 & 3 & 0.713 \\
            & 16 & 4 & 0.520 \\
            & 20 & 3 & 0.746 \\
            & 20 & 4 & 0.521 \\ \hline
            \multirow{10}{*}{saj12} & 6  & 3 & 4.111 \\
            & 6  & 4 & 3.578 \\
            & 8  & 3 & 3.409 \\
            & 8  & 4 & 3.054 \\
            & 12 & 3 & 3.267 \\
            & 12 & 4 & 2.638 \\
            & 16 & 3 & 3.021 \\
            & 16 & 4 & 2.783 \\
            & 20 & 3 & 2.703 \\
            & 20 & 4 & 2.849 \\ \hline
        \end{tabular}
        \caption{MPI algoritmus.}
        \label{tab:mpi}
    \end{table}

    \subsection{Zrychlení}

    \begin{table}[h]
        \begin{subtable}{.5\linewidth}\centering
        \begin{tabular}{|c|c|c|l|l|l|l|l|l|}
            \hline
            \textbf{instance / p} &
            \textbf{1} &
            \textbf{2} &
            \multicolumn{1}{c|}{\textbf{4}} &
            \multicolumn{1}{c|}{\textbf{6}} &
            \multicolumn{1}{c|}{\textbf{8}} &
            \multicolumn{1}{c|}{\textbf{10}} &
            \multicolumn{1}{c|}{\textbf{16}} &
            \multicolumn{1}{c|}{\textbf{20}} \\ \hline
            saj7 &
                {\color[HTML]{FE0000} 18.094} &
                {\color[HTML]{FE0000} 14.589} &
                {\color[HTML]{FE0000} 259.065} &
                {\color[HTML]{FE0000} 60.309} &
                {\color[HTML]{FE0000} 96.200} &
                {\color[HTML]{FE0000} 227.303} &
                {\color[HTML]{FE0000} 1098.935} &
                {\color[HTML]{FE0000} 686.489} \\ \hline
            saj10 & 0.785 & 0.520 & 0.832 & 1.019 & 1.210 & 1.601 & 1.201 & 1.064 \\ \hline
            saj12 & 0.854 & nan   & 1.295 & 0.969 & 1.006 & 1.373 & nan   & 1.444 \\ \hline
        \end{tabular}
        \caption{taskový paralelismus} \label{tab:speedup-openmp-task}
        \end{subtable}%
        \bigskip
        \bigskip

        \begin{subtable}{.5\linewidth}\centering
        \begin{tabular}{|c|c|c|l|l|l|l|l|l|}
            \hline
            \textbf{instance / p} &
            \textbf{p=1} &
            \textbf{p=2} &
            \multicolumn{1}{c|}{\textbf{4}} &
            \multicolumn{1}{c|}{\textbf{6}} &
            \multicolumn{1}{c|}{\textbf{8}} &
            \multicolumn{1}{c|}{\textbf{10}} &
            \multicolumn{1}{c|}{\textbf{16}} &
            \multicolumn{1}{c|}{\textbf{20}} \\ \hline
            saj7  & 0.872 & 0.920 & 0.920 & 0.920 & 5.857 & {\color[HTML]{FE0000} 787.676} & {\color[HTML]{FE0000} 644.293} & {\color[HTML]{FE0000} 644.293} \\ \hline
            saj10 & 0.850 & 0.586 & 0.810 & 0.951 & 1.036 & 1.052                          & 1.025                          & 1.263                          \\ \hline
            saj12 & 0.849 & 0.849 & 1.178 & 1.178 & 1.178 & 1.178                          & 1.178                          & 1.178                          \\ \hline
        \end{tabular}
        \caption{datový paralelismus} \label{tab:speedup-openmp-data}
        \end{subtable}%
        \bigskip
        \bigskip

        \begin{subtable}{.5\linewidth}\centering
        \begin{tabular}{|c|c|c|l|l|l|l|l|l|l|l|}
            \hline
            \textbf{instance / (p, n)} &
            \textbf{(6,3)} &
            \textbf{(6,4)} &
            \multicolumn{1}{c|}{\textbf{(8,3)}} &
            \multicolumn{1}{c|}{\textbf{(8,4)}} &
            \multicolumn{1}{c|}{\textbf{12,3)}} &
            \multicolumn{1}{c|}{\textbf{(12,4)}} &
            \multicolumn{1}{c|}{\textbf{(16,3)}} &
            \multicolumn{1}{c|}{\textbf{(16,4)}} &
            \multicolumn{1}{c|}{\textbf{(20,3)}} &
            \multicolumn{1}{c|}{\textbf{(20,4)}} \\ \hline
            saj7  & 2.402 & 2.787 & 2.992 & 3.445 & 3.450 & 3.752 & 3.509 & 3.978 & 3.577 & 3.754 \\ \hline
            saj10 & 1.894 & 2.960 & 2.193 & 3.187 & 2.190 & 3.440 & 2.420 & 3.321 & 2.313 & 3.313 \\ \hline
            saj12 & 1.986 & 2.281 & 2.394 & 2.673 & 2.499 & 3.094 & 2.702 & 2.933 & 3.020 & 2.865 \\ \hline
        \end{tabular}
        \caption{MPI}\label{tab:speedup-mpi}
        \end{subtable}%

        \caption{Naměřené zrychlení. Počet výpočetních vláken je označen jako p, počet uzlů jako n (pouze u MPI). Barevně zvýrazněné je superlineární zrychlení.}
        \label{tab:speedup}
    \end{table}

    \subsection{Efektivita}

    \begin{table}[h]
        \begin{subtable}{.5\linewidth}\centering
        \begin{tabular}{|c|c|c|l|l|l|l|l|l|}
            \hline
            \textbf{instance / p} &
            \textbf{1} &
            \textbf{2} &
            \multicolumn{1}{c|}{\textbf{4}} &
            \multicolumn{1}{c|}{\textbf{6}} &
            \multicolumn{1}{c|}{\textbf{8}} &
            \multicolumn{1}{c|}{\textbf{10}} &
            \multicolumn{1}{c|}{\textbf{16}} &
            \multicolumn{1}{c|}{\textbf{20}} \\ \hline
            saj7 &
                {\color[HTML]{FE0000} 18.094} &
                {\color[HTML]{FE0000} 7.294} &
                {\color[HTML]{FE0000} 64.766} &
                {\color[HTML]{FE0000} 10.051} &
                {\color[HTML]{FE0000} 12.025} &
                {\color[HTML]{FE0000} 22.730} &
                {\color[HTML]{FE0000} 68.683} &
                {\color[HTML]{FE0000} 34.324} \\ \hline
            saj10 & 0.785 & 0.260 & 0.208 & 0.170 & 0.151 & 0.160 & 0.075 & 0.053 \\ \hline
            saj12 & 0.854 & nan   & 0.324 & 0.161 & 0.126 & 0.137 & nan   & 0.072 \\ \hline
        \end{tabular}
        \caption{taskový paralelismus}
        \label{tab:efektivita-openmp-task}
        \end{subtable}%

        \begin{subtable}{.5\linewidth}\centering
        \begin{tabular}{|c|c|c|l|l|l|l|l|l|}
            \hline
            \textbf{instance / p} &
            \textbf{1} &
            \textbf{2} &
            \multicolumn{1}{c|}{\textbf{4}} &
            \multicolumn{1}{c|}{\textbf{6}} &
            \multicolumn{1}{c|}{\textbf{8}} &
            \multicolumn{1}{c|}{\textbf{10}} &
            \multicolumn{1}{c|}{\textbf{16}} &
            \multicolumn{1}{c|}{\textbf{20}} \\ \hline
            saj7  & 0.872 & 0.460 & 0.497 & 0.802 & 0.732 & {\color[HTML]{FE0000} 78.768} & {\color[HTML]{FE0000} 40.268} & {\color[HTML]{FE0000} 45.575} \\ \hline
            saj10 & 0.850 & 0.293 & 0.202 & 0.158 & 0.130 & 0.105                         & 0.064                         & 0.063                         \\ \hline
            saj12 & 0.849 & 0.496 & 0.295 & 0.231 & 0.159 & 0.122                         & 0.107                         & 0.045                         \\ \hline
        \end{tabular}
        \caption{datový paralelismus}
        \label{tab:efektivita-openmp-data}
        \end{subtable}%

        \begin{subtable}{.5\linewidth}\centering
        \centering
        \begin{tabular}{|c|c|c|l|l|l|l|l|l|l|l|}
            \hline
            \textbf{instance / (p,n)} &
            \textbf{(6,3)} &
            \textbf{(6,4)} &
            \multicolumn{1}{c|}{\textbf{(8,3)}} &
            \multicolumn{1}{c|}{\textbf{(8,4)}} &
            \textbf{12,3)} &
            \textbf{(12,4)} &
            \textbf{(16,3)} &
            \textbf{(16,4)} &
            \textbf{(20,3)} &
            \textbf{(20,4)} \\ \hline
            saj7  & 0.133 & 0.116 & 0.125 & 0.108 & 0.096 & 0.078 & 0.073 & 0.062 & 0.060 & 0.047 \\ \hline
            saj10 & 0.105 & 0.123 & 0.091 & 0.100 & 0.061 & 0.072 & 0.050 & 0.052 & 0.039 & 0.041 \\ \hline
            saj12 & 0.110 & 0.095 & 0.100 & 0.084 & 0.069 & 0.064 & 0.056 & 0.046 & 0.050 & 0.036 \\ \hline
        \end{tabular}
        \caption{MPI}
        \label{tab:efektivita-mpi}
        \end{subtable}

        \caption{Naměřená efektivita. Počet výpočetních vláken je označen jako p, počet uzlů jako n (pouze u MPI). Barevně zvýrazněná je efektivita větší než 1.}
        \label{tab:efektivita}
    \end{table}

\end{document}
